\documentclass[12pt]{article}
\usepackage[a4paper, margin=1in]{geometry}
\usepackage{mathptmx}
\begin{document}

\begin{center}
\Large Summary and Review of: Gate-Level Netlist Reverse Engineering for
Hardware Security: Control Logic Register Identification\\
\normalsize Brian Weber
\end{center}
\section{Abstract}
It is easy and relatively inexpensive to obtain a full netlist from fabricated
chips using reverse engineering, however there is still a lack of tools to
analyze this netlist to look for malicious functionality. This paper introduces
a solution for analyzing and categorizing the funtionality of registers given a
netlist called RELIC (Reverse Engineering Logic Identification Classification).
Using the methods described in the paper control logic registers were able to be
distinguished from data registers in circuits of varying complexity.

\section{Methods}
Given an arbitrary netlist, RELIC takes an arbitrary netlist and produces
assigns scores called "Similarity Scores" to register gate pairs by using
Dynamic Programming and graph algorithms. Based on these scores, the registers
in a netlist are more easily classified, and it is easier to distinguish between
real and malicious registers. RELIC assumes register pairs from the same
datapath have similar deep logical structures. Using this assumption, RELIC
analyzes and compares the fan-in structure of each register to generate
similarity scores. Using this technique, logic registers are found by finding
registers that are not part of a data path. In addition, trojan logic registers
are identified based on the assumption that it is unlikely they will have
similar logic to valid logic registers.

To generate these scores, RELIC first reduces all input structures to AND-OR-INV
logic (converting xor to 2 ands and an or, nand to and+inv etc.). To solve
obfuscation (accidental or intentional), logic gates are merged until they
cannot be merged anymore. No gates are merged that are used for registers. After
preprocessing, similarity scores are generated based on the topological
structure of the netlist. Scores are generated for arbitrary pairs of logic
vertexes from 0 to 1. A score of 0 means the fan-in structures of the 2 
registers have no similarities, while a score of 1 means they are identical. If 
the score exceeds some predetermined threshhold it is added to a bipartate graph.    
memoization, results are cached to prevent recomputation of 2 vertexes with the
same value. This process is done recursively up to a predtermined depth. After
the scores are calculated, registers are classified. Registers that are similar
to many other registers are typically not logic affecting registers. 

\section{Results}
RELIC seems to be very good at classifying registers, and based on the tests
done, did so efficiently. At a minimum, it correctly classified 89.1\% of
registers. It was also able to classify registers in a circuit with 12576 gates
with 3968 registers with 100\% accuracy in 4 minutes. Circuits that have higher
logical complexity showed that they needed lower depths to be computed
efficiently. This in turn negatively affected the accuracy of the computation,
causing the similarity scores to be higher than they should have.

\section{Discussion}

This seems to be a strong option for detecting malicious registers given a
netlist. Combining this with the controllability and observability analysis
described in the previous provides a robust set of tools for detecting
malicious structure in netlists, though I am not completely sure what this
method provides that the controllability and observability analysis does not
provide. This method seems to be very efficient for circuits of low logic
complexity, but as the MC8051 example shows, for circuits with higher logical
complexity, it becomes more difficult to compute the similarity scores. This
makes me worry about how this method scales to large and logically complex
circuits. In addition, I would have liked to see more tests done on many more
and larger circuits. Finally, while accuracy was mostly good, it was
inconsistent depending on the structure of the design, which may mean this
inconsitency might be able to be abused by an adversary. 

\end{document}
